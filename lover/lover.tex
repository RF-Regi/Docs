\documentclass[a4paper,11pt,norsk]{scrartcl}
% generated by Docutils <https://docutils.sourceforge.io/>
\usepackage{cmap} % fix search and cut-and-paste in Acrobat
\usepackage{ifthen}
\usepackage[T1]{fontenc}
\usepackage[utf8]{inputenc}
\usepackage[norsk]{babel}
\setcounter{secnumdepth}{0}
\usepackage{textcomp} % text symbol macros

%%% Custom LaTeX preamble
% PDF Standard Fonts
\usepackage{mathptmx} % Times
\usepackage[scaled=.90]{helvet}
\usepackage{courier}

%%% User specified packages and stylesheets

%%% Fallback definitions for Docutils-specific commands

% subtitle (in document title)
\providecommand*{\DUdocumentsubtitle}[1]{{\large #1}}

% transition (break / fancybreak / anonymous section)
\providecommand*{\DUtransition}{%
  \hspace*{\fill}\hrulefill\hspace*{\fill}
  \vskip 0.5\baselineskip
}

% hyperlinks:
\ifthenelse{\isundefined{\hypersetup}}{
  \usepackage[colorlinks=true,linkcolor=blue,urlcolor=blue]{hyperref}
  \usepackage{bookmark}
  \urlstyle{same} % normal text font (alternatives: tt, rm, sf)
}{}
\hypersetup{
  pdflang={no},
  pdftitle={RF-REGIS LOVER},
}

%%% Body
\begin{document}
\title{RF-REGIS LOVER%
  \label{rf-regis-lover}%
  \\%
  \DUdocumentsubtitle{Vedtatt 27. April 2022}%
  \label{vedtatt-27-april-2022}}
\author{}
\date{}
\maketitle

\phantomsection\label{contents}
\pdfbookmark[1]{Contents}{contents}
\tableofcontents

%___________________________________________________________________________
\DUtransition


\section{FORMÅL OG ORGANISASJON%
  \label{formal-og-organisasjon}%
}


\subsection{§ 1 TILHØRIGHET%
  \label{tilhorighet}%
}

RF-Regi er en undergruppe av Realistforeningen, fakultetsforening ved Det
matematisk-naturvitenskapelig fakultetet, Universitet i Oslo


\subsection{§ 2 FORMÅL%
  \label{formal}%
}

RF-Regis formål er å drifte og vedlikeholde lyd- og lyssystemer, samt
bedrive utleie av disse til de andre foreningene ved Universitetet i Oslo.


\subsection{§ 3 UTBYTTE%
  \label{utbytte}%
}

RF-Regi er en ideell organisasjon som ikke betaler ut utbytte.
RF-Regi skal over et lenger tidsperspektiv ikke gå med noe overskudd.
Overskuddene går inn i egenkapitalen til foreningen, og brukes
til innkjøp av nytt utstyr.


\subsection{§ 4 MEDLEMSKAP%
  \label{medlemskap}%
}

RF-Regi og Realistforeningen har felles medlemskap, så fremt det enkelte medlem
av Realistforeningen samtykker til dobbelt medlemskap.

Medlemskap i RF-regi tilbys alle medlemmer i Realistforeningen.


\subsection{§ 5 OPPLØSNING%
  \label{opplosning}%
}

Går en generalforsamling inn for å oppløse foreningen, skal
desisjonsutvalget tidligst tre uker og senest fire måneder senere
innkalle til ekstraordinær generalforsamling hvor saken skal behandles
på nytt. Dersom foreningen oppløses, disponerer Realistforeningen
over foreningens aktiva.


\subsection{§ 6 VED OPPLØSNING%
  \label{ved-opplosning}%
}

\begin{enumerate}
\renewcommand{\labelenumi}{\alph{enumi})}
\item Ved oppløsning av foreningen overføres alle eiendeler til
Realistforeningen, og Realistforeningen  tar over utleievirksomheten av
utstyret.

\item Dersom Realistforeningen ikke ønsker å ta over utleievirksomheten skal
Realistforeningen kalle inn til et møte hvor kjellerforeningene Samfunnsvitenskaplig
fakultetsforening, Medicinerforeningen, Kjellerutvalget, Filologisk
forening/Uglebo og Cybernetisk selskab skal
bli enige om en omfordeling av utstyret. Om kjellerforeningene ikke blir enige,
tilfaller utstyret Kulturstyret ved Velferdstinget i Oslo og Akershus, som
omfordeler utstyret blant kjellerforeningene etter eget forgodtbefinnende.
Utstyrets verdi regnes etter prinsippet lineær årlig avskrivning. Utstyret
kan ikke regnes mot eventuelle utestående fordringer foreningene har hos RF-Regi.
\end{enumerate}


\subsection{§ 7 TAP AV UTSTYR%
  \label{tap-av-utstyr}%
}

RF\-Regi plikter å informere kjellerforeningene om eventuelle tap av hele eller
deler av utstyret kjøpt inn med Kulturstyrets midler innen rimelig tid.
Eventuelle tap kan kreves dokumentert av en eller flere av kjellerforeningene.


\subsection{§ 8 STYRET%
  \label{styret}%
}

\begin{enumerate}
\renewcommand{\labelenumi}{\alph{enumi})}
\item Styret i RF-Regi består av: regiformann, forretningsfører, utleieansvarlig,
DJ-sjef og Realistforeningens formann.

\item Regiformann velges på vårens generalforsamling og sitter for ett år.
Regiformann er foreningens daglige leder og styreleder.

\item Forretningsfører velges på høstens generalforsamling og sitter for ett år.
Forretningsfører er ansvarlig for foreningens økonomi og er foreningens nestleder.

\item Utleieansvarlig velges på generalforsamling og sitter for ett semester.
Utleieansvarlig har overordnet ansvar for drift av RF-Regis utleievirksomhet.

\item DJ-sjef velges på generalforsamling og sitter for ett semester. DJ-sjef er
ansvarlig for daglig drift av DJ-gruppen.

\item Realistforeningens formann er Realistforeningens representant i styret.

\item Regiformann og forretningsfører tegner sammen for foreningen.

\item Styrets oppgave er å koordinere langsiktig virksomhet, vedlikeholde utstyr,
godkjenne innkjøp og representere foreningen utad.

\item Styret kan tilknytte seg det antall funksjonærer styret finner nødvendig.
\end{enumerate}


\subsection{§ 9 ERFARINGSOVERFØRING%
  \label{erfaringsoverforing}%
}

Alle tillitsvalgte valgt av
Generalforsamlingen og innehavere av verv oppnevnt av styret
ihht. § 8 a, har ansvar for å lage og oppdatere erfaringsprotokoller
for opplæring av etterfølgende innehavere av vervet og funksjonærer
tilknyttet styret. Alle personer omfattet av første punktum
har ansvar for at påtroppende vervinnehaver får den nødvendige
opplæring for å inneha sitt verv.


\subsection{§ 10 ØKONOMI%
  \label{okonomi}%
}

\begin{enumerate}
\renewcommand{\labelenumi}{\alph{enumi})}
\item Forretningsfører har ansvaret for RF-Regis regnskap og for
å lære opp påtroppende vervinnehaver i økonomistyring.

\item Forretningsfører fører regnskapene ut den inneværende regnskapsperiode.

\item Realistforeningen overtar driften av RF-Regi inntil nytt
styre er valgt dersom det sittende ikke kan funksjonere. Ingen
utbetalinger, med unntak av utestående fordringer, skal skje før en
generalforsamling er avholdt.
\end{enumerate}


\subsection{§ 11 DESISJONSUTVALGET%
  \label{desisjonsutvalget}%
}

\begin{enumerate}
\renewcommand{\labelenumi}{\alph{enumi})}
\item RF-Regi er en undergruppe av Realistforeningen, og derfor underlagt dets
Desisjonsutvalg.

\item Desisjonsutvalget har den endelige avgjørelsen i tvilsspørsmål om
tolkning av lovene. Utvalget kan også fatte vedtak og gi regler i
situasjoner hvor lovene måtte vise seg å være utilstrekkelige.
Ethvert medlem av RF-Regi har rett til å innanke for
Desisjonsutvalget vedtak hvor det kan være tvil om lovligheten.
\end{enumerate}


\subsection{§ 12 REVISJONSUTVALGET%
  \label{revisjonsutvalget}%
}

\begin{enumerate}
\renewcommand{\labelenumi}{\alph{enumi})}
\item RF-Regi er en undergruppe av Realistforeningen, og derfor underlagt dets
Revisjonsutvalg.

\item Regnskap skal være innlevert senest tre uker før generalforsamling påfølgende
semester. Blir ikke regnskapene godkjent på generalforsamling, skal
regiforfører inndra alle bilag og fullføre regnskapet. Det kan gis dispensasjon
til avvik fra dette punktet av styret i samarbeid med Revisjonsutvalget og forretningsfører.

\item På Generalforsamlingen skal Revisjonsutvalget legge frem revisjonsberetningen,
som skal være skrevet av Revisjonsutvalget selv eller, hvis styret finner det
nødvendig, en registrert eller statsautorisert revisor. Revisjonsutvalet
har ansvar for å opplyse Generalforsamlingen om eventuelle budsjettsoverskridelser.
\end{enumerate}


\subsection{§ 13 GENERALFORSAMLING%
  \label{generalforsamling}%
}

\begin{enumerate}
\renewcommand{\labelenumi}{\alph{enumi})}
\item Generalforsamlingen er foreningens høyeste myndighet i spørsmål som
ikke kommer inn under § 11 b.  Alle medlemmer av
RF-Regi har tale- og forslagsrett.

Alle medlemmer av RF-Regi som var innmeldt før innkallingen ble
offentliggjort har stemmerett på generalforsamlingen.

Generalforsamlingen er beslutningsdyktig når minst 1/10 av de
stemmeberettigede er til stede. Samtidig er det tilstrekkelig med 50
stemmeberettigede på generalforsamlingen dersom foreningen har mer
enn 500 medlemmer.

\item Ordinær generalforsamling avholdes etter Realistforeningens
generalforsamling. Ekstraordinær generalforsamling avholdes når styret
vedtar det eller det kreves av Desisjonsutvalget eller minst 1/10
av medlemmene, dog slik at 50 medlemmer er tilstrekkelig dersom
foreningen har mer enn 500 medlemmer.

\item Innkallelse til ordinær og ekstraordinær generalforsamling må være
offentliggjort minst 10 virkedager i forveien. Ved ordinær og
ekstraordinær generalforsamling må forslag til foreløpig dagsorden
være offentliggjort senest 5 virkedager i forveien. Som virkedag
regnes alle dager i samme semester som ikke er helg, offentlig
høytidsdag eller feriedag for studentene ved Det matematisk-
naturvitenskapelige fakultet i henhold til fakultetets offisielle
kalender. Generalforsamlinger innkalles av styret. Dersom
dette ikke fungerer eller ikke etterkommer lovlige krav om at
generalforsamling skal kalles inn, skal Desisjonsutvalget overta
styrets plikter når det gjelder generalforsamlinger.

\item Forslag om lovendringer og andre saker som ønskes tatt opp på
generalforsamlingen må være styret i hende og offentliggjøres
5 virkedager før. Desisjonsutvalget kan fremme endringsforslag
inntil 48 timer før generalforsamlingen. Lovendringsforslag kan
ikke behandles på ekstraordinær generalforsamling.

\item Generalforsamlingen kan foreta endringer i rekkefølgen av punktene
til det endelige forslag til dagsorden. Den kan også utelukke ett
eller flere av de foreslåtte punkter så lenge det ikke strider mot
§ 13 k. Den endelige dagsorden godkjennes av
generalforsamlingen. I forbindelse med godkjennelse av dagsorden
skal det velges ordstyrer, referent og to medlemmer til å
undertegne generalforsamlingens protokoll.

\item Ethvert medlem kan på generalforsamlingen foreslå tatt opp saker
utenom den oppsatte dagsorden. Generalforsamlingen kan ikke fatte
vedtak i slike saker.

\item Generalforsamlingen kan med alminnelig flertall gi ikke-medlemmer
møte- og talerett.

\item Avstemninger på generalforsamlinger skal være hemmelige når minst
tre stemmeberettigede krever det.

\item Valgbare som tillitsvalgte i RF-Regi er alle foreningens medlemmer,
med unntak av medlemer som sitter i Realistforeningens Økonomiutvalg,
Revisjonsutvalg eller Desisjonsutvalg.

\item Valg på flere tillitsvalgte under ett avgjøres med alminnelig
flertall. Ved valg på en enkelt tillitsvalgt kan tre
stemmeberettigede kreve at valget skal avgjøres med absolutt
flertall. Oppnår ingen dette ved første avstemming, avholdes bundet
omvalg.

\item På ordinær generalforsamling behandles:

\begin{enumerate}
\renewcommand{\labelenumii}{\arabic{enumii}.}
\item Regnskap, etter en redegjørelse for RF-Regis totale økonomi.

\item Budsjettrammer. På høstens generalforsamling vedtas
budsjettrammer for neste år. På vårens generalforsamling kan
budsjettrammene revideres.

\item Eventuelle lovendringsforslag

\item Semesterberetning

\item Valg av tillitsvalgte:

\begin{enumerate}
\renewcommand{\labelenumiii}{\arabic{enumiii})}
\item Regiformann (§8b)

\item Forretningsfører (§8c)

\item Utleieansvarlig (§8d)

\item DJ-sjef (§8e)
\end{enumerate}
\end{enumerate}
\end{enumerate}


\subsection{§ 14 MISTILLIT%
  \label{mistillit}%
}

Foreningens medlemmer kan fremme mistillitsforslag mot tillitsvalgte
som er valgt ihht. § 13 k punkt 5. Slike forslag kan bare behandles
av en generalforsamling, og må være fremmet 48 timer før
generalforsamlingen. Mistillitsforslag vedtas med 2/3 flertall. Dersom mistillitsforslaget
mot et medlem av styret blir vedtatt, kan generalforsamlingen vedta å
holde nyvalg på samtlige medlemmer av styret for resten av hvert
medlems funksjonstid.


\subsection{§ 15 LOVENDRINGER%
  \label{lovendringer}%
}

Forslag til lovendring skal bare behandles på ordinær
generalforsamling, og må få 2/3 flertall blant de tilstedeværende
stemmeberettigede for å vedtas.


\subsection{§ 16 LOVERS GYLDIGHET%
  \label{lovers-gyldighet}%
}

Disse lovene er gyldige fra den dag de blir vedtatt, slik at alle
tidligere lover opphører å gjelde fra samme dag.

\end{document}
